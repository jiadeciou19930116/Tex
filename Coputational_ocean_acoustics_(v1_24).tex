\documentclass[journal,onecolumn]{IEEEtran}
%\documentclass[journal]{IEEEtran}
\usepackage{color}

\begin{document}

\title{Parabolic Equation}         
\author{Jia-De Ciou and Jean-Fu Kiang}        
\date{}          

\markboth{}{}

\maketitle

\begin{abstract}

\end{abstract}

\textcolor{red}{Do not erase red words, use blue words for changes.}
\definecolor{dg}{rgb}{0, 0.5, 0}
\newcommand{\sech}{\ensuremath{\mathrm{sech\ }}}

\begin{IEEEkeywords}
Wave Equation
\end{IEEEkeywords}



\section{Standard Parabolic Equations}

The acoustic pressure $p$ in an ideal fluid satisfies the wave equation (Helmholtz equation) as \cite{COA} 
\begin{eqnarray}
&&\rho \nabla \cdot \left (\frac{1}{\rho} \nabla p(\bar{r}) \right) + k^2 p(\bar{r}) = - \frac{\delta(r)\delta(z-z_s)}{2 \pi r} 
\label{eqn_0001}
\end{eqnarray}
where 
\begin{eqnarray}
&&k^2 = \frac{\omega^2}{c^2(\bar{r})} 
\label{eqn_0002}
\end{eqnarray}
$\rho$ is medium's density, $k$ is the wavenumber, $\delta(r)$ and $\delta(z-z_s)$ is Dirac delta function, $c(\bar{r})$ is the sound speed in the medium, and $\omega$ is the angular frequency.
If the fluid is homogeneous, than (\ref{eqn_0001}) can be simplifed as
\begin{eqnarray}
&&\left( \nabla^2 + k^2 \right) p( \bar{r} ) = 0 
\label{eqn_0003}
\end{eqnarray}


The Laplacian operator can be represented in cylindrical coordinates $(r, \phi, z)$ as
\begin{eqnarray}
&&\nabla^2 = \frac{1}{r} \frac{\partial}{\partial r} r \frac{\partial}{\partial r} 
+ \frac{1}{r^2} \frac{\partial^2}{\partial \phi^2}
+ \frac{\partial^2}{\partial z^2}   
\label{eqn_0004}
\end{eqnarray}

If the pressure field is azimuthally symmetric, $\partial / \partial \phi = 0$, 
thus (\ref{eqn_0003}) reduces to a 2D Helmholtz equation as
\begin{eqnarray}
&&\frac{\partial^2 p}{\partial r^2} + \frac{1}{r}\frac{\partial p}{\partial r} 
+ \frac{\partial^2 p}{\partial z^2} + k^2_0 n^2 p = 0 
\label{eqn_0005}
\end{eqnarray}
where $c_0$ is a reference speed of sound, $k_0 = \omega / c_0$ is the reference wavenumber in the medium at radial frequency $\omega$, 
$n(r, z) = c_0 / c(r, z) = k / k_0$ is the refraction index.

\subsection{Standard PE Derivation}

Use Tappert's way\cite{Tappert_WPU_tpam} to get standard form of 2D parabolic wave equation by assuming the solution of (\ref{eqn_0005}) 
to take the Helmholtz equation form
\begin{eqnarray}
&&p(r, z) = \psi(r, z) H^{(1)}_0 (k_0 r) 
\label{eqn_0006}
\end{eqnarray}
where $H^{(1)}_0 (k_0 r)$ is the Hankel function, representing an outgoing cylindrical wave, which satisfies
\begin{eqnarray}
&&\frac{\partial^2 H^{(1)}_0(k_0 r)}{\partial r^2} 
+ \frac{1}{r} \frac{\partial H^{(1)}_0(k_0 r)}{\partial r} + k^2_0 H^{(1)}_0 (k_0 r) = 0 
\label{eqn_0007}
\end{eqnarray}
The envelope function $\psi(r, z)$ is assumed to vary slowly.

Under far-field approximation, $k_0 r \gg 1$, the Hankel function is approximated as
\begin{eqnarray}
&&H^{(1)}_0 (k_0 r) \simeq \sqrt{\frac{2}{\pi k_0 r}} e^{i (k_0 r - \pi / 4)}
\label{eqn_0008}
\end{eqnarray}
and
\begin{eqnarray}
&&p(r, z) \simeq \psi(r, z) \sqrt{\frac{2}{\pi k_0 r}} e^{i (k_0 r - \pi / 4)}
\label{eqn_0009}
\end{eqnarray}
Substituting (\ref{eqn_0006}) into (\ref{eqn_0005}), and making use of the Hankel-function property given by (\ref{eqn_0007}), we obtain
\begin{eqnarray}
&&\frac{\partial ^2 \psi}{\partial r ^2} 
+ \left( \frac{2}{H_0^{(1)}}\frac{\partial H_0^{(1)}}{\partial r} + \frac{1}{r} \right) \frac{\partial \psi}{\partial r} 
+ \frac{\partial ^2 \psi}{\partial z ^2}+ k_0^2 (n^2 - 1) \psi = 0
\label{eqn_0010}
\end{eqnarray}
Next, we make the far-field assumption, $k_0 r \gg 1$, use (\ref{eqn_0008}) and ignore $1/r$ term to obtain the simplified elliptic wave equation
\begin{eqnarray}
&&\frac{\partial^2 \psi}{\partial r^2} 
+ 2ik_0\frac{\partial \psi}{\partial r} 
+ \frac{\partial^2 \psi}{\partial z^2} + k_0^2 (n^2 - 1) \psi = 0
\label{eqn_0011}
\end{eqnarray}
Assume that $\psi$ varies slowly over a range of one a wavelengh $\lambda$,
$\partial \psi / \partial r \ll \psi / \lambda \sim k_0\psi$.
Thus, the first term in (\ref{eqn_0011}) can be ignored to have
\begin{eqnarray}
&&2ik_0\frac{\partial \psi}{\partial r} 
+ \frac{\partial^2 \psi}{\partial z^2} 
+ k_0^2 (n^2 - 1) \psi = 0
\label{eqn_1012}
\end{eqnarray}
which is a standard parabolic equation suitable for underwater acoustics\cite{HT_App_SSF}.
However, (\ref{eqn_1012}) is accurate only at propagation angle smaller than 10-$15^\circ$ off the horizontal.

\subsection{Generalized PE Derivation}

To derive a series of parabolic wave equations for different propagation angles, 
the first step is to define two operators\cite{DWC_NORDA}
\begin{eqnarray}
&&P = \frac{\partial}{\partial r}, \hspace{0.5cm} 
Q = \sqrt{n^2 + \frac{1}{k^2_0} \frac{\partial^2}{\partial z^2}}  
\label{eqn_1013}
\end{eqnarray}
and write (\ref{eqn_0011}) as
\begin{eqnarray}
&&[ P^2 + 2ik_0 P + k^2_0 (Q^2 - 1) ] \psi = 0 
\label{eqn_1014}
\end{eqnarray}
which is factored into an outgoing and an incoming wave components as 
\begin{eqnarray}
&&( P + i k_0 - i k_0 Q )( P + i k_0 + i k_0 Q ) \psi - i k_0 ( P Q - Q P) \psi = 0 
\label{eqn_1015}
\end{eqnarray}
If the media is range independent, namely, $n = n(z)$, then
\begin{eqnarray}
&&PQ \psi = QP \psi
\end{eqnarray}
We assume that the range dependence in $n(r, z)$ is weak enough that the last term in (\ref{eqn_1015}) can also be ignored.
For an outgoing wave component,
\begin{eqnarray}
&&P \psi = i k_0(Q - 1) \psi 
\label{eqn_1016}
\end{eqnarray}
or
\begin{eqnarray}
&&\frac{\partial \psi}{\partial r} = i k_0\left(\sqrt{n^2 + \frac{1}{k^2_0} \frac{\partial^2}{\partial z^2}} - 1 \right) \psi 
\label{eqn_1016.1}
\end{eqnarray}
which is a one-way wave equation, and the solution can be represented as
\begin{eqnarray}
&&\psi(r, z) = \psi_0 (z) \exp \left\{ i k_0 (r - r_0) \left( \sqrt{n^2 + \frac{1}{k^2_0} \frac{\partial^2}{\partial z^2}} - 1 \right)
\right\}
\label{eqn_1017}
\end{eqnarray}
where $\psi_0 (z)$ is a reference of $\psi(r, z)$ at $r = r_0$.

\subsection{Expansion of the Square-Root Operator}

The $Q$ operator can be represented as
\begin{eqnarray}
&&Q = \sqrt{1 + q}
\label{eqn_1017.1}
\end{eqnarray}
where
\begin{eqnarray}
&&q = \varepsilon + \mu, \qquad \varepsilon = n ^2 - 1, \qquad \mu = \frac{1}{k_0^2}\frac{\partial^2}{\partial z^2}
\label{eqn_1017.2}
\end{eqnarray}
Then, Eq.(\ref{eqn_1017}) can be rewritten as
\begin{eqnarray}
&&\psi(r, z) = \psi_0(z) \exp \left\{ i k_0 (r - r_0) (Q - 1) \right\}
\label{eqn_1016.2}
\end{eqnarray}

The Taylor's series expansion of $Q$ is represented as
\begin{eqnarray}
&&Q = 1 + \frac{q}{2} - \frac{q^2}{8} + \frac{q^3}{16} + \cdots
\label{eqn_1018}
\end{eqnarray}
with the requirement of $|q| < 1 $. 

Consider a trial plane-wave solution of the form
\begin{eqnarray}
&&\psi(r, z) = e^{i ( k_r r \pm k_z z )} 
\label{eqn_1019} 
\end{eqnarray}
under the dispersion relation $k^2 = k_r^2 + k_z^2$, 
where $k_r$ and $k_z$ are the horizontal and vertical wavenumbers, respectively.
The propagation angle $\theta$ can be expressed as 
\begin{eqnarray}
&&\sin \theta = \pm \frac{k_z}{k}
\label{eqn_1020}
\end{eqnarray}

By applying the differential operator $\mu$ to (\ref{eqn_1019}), we have
\begin{eqnarray}
&&\mu \psi(r, z) = \frac{1}{k^2_0} \frac{\partial^2}{\partial z^2}\psi(r, z) = - \frac{k_z^2}{k_0^2} \psi(r, z) 
= - n^2 \sin^2 \theta  \psi(r, z)
\nonumber
\end{eqnarray}
implying that
\begin{eqnarray}
&&\mu = - n^2 \sin^2 \theta = - k_z^2 / k_0^2
\label{eqn_1021}
\end{eqnarray}

The Snell's law states that
\begin{eqnarray}
&&\frac{\cos \theta_0}{\cos \theta} = \frac{c_0}{c} = n
\label{eqn_10211}
\end{eqnarray}
Thus, (\ref{eqn_1017.2}) implies that
\begin{eqnarray}
&&q = (n^2 - 1) - n^2 \sin^2 \theta = - \sin^2 \theta_0
\label{eqn_1022}
\end{eqnarray}
which means when the sound at source is $c_0$, $q$ will be a function of the propagation angle at source.
An expansion of $\sqrt{1 + q}$ around $q = 0$ is equivalent to introducing a paraxial approximation.
By retaining only the first two terms in (\ref{eqn_1018}), we get the approximate form of the square-root operator,
\begin{eqnarray}
&&Q \simeq 1 + \frac{q}{2} = 1 + \frac{n^2 - 1}{2} + \frac{1}{2k_0^2}\frac{\partial^2}{\partial z^2}
\label{eqn_1023}
\end{eqnarray}

which is substituted into (\ref{eqn_1016}) to obtain
\begin{eqnarray}
\frac{\partial \psi}{\partial r} = \frac{ik_0}{2} \left(n^2 - 1 + \frac{1}{k_0^2}\frac{\partial^2 }{\partial z^2}\right)\psi
\label{eqn_1024}
\end{eqnarray}
which is the same as in (\ref{eqn_1012}).

In principle, retaining more high-order terms in the Taylor's series, at the cost of more complicated numerical implementation,
will lead to more accurate solution in a wider angle-range.
A general form of approximations can be represented as
\begin{eqnarray}
&&\sqrt{1 + q} \simeq \frac{a_0 + a_1 q}{b_0 + b_1 q}
\label{eqn_1025}
\end{eqnarray}
where the coefficients are chosen to minimize the error over a given angle interval. 
Three sets of such coefficients are listed below.
\begin{eqnarray}
&&\sqrt{1 + q} \simeq 1 + 0.5 q, \hskip 3mm {\rm Tappert} 
\label{eqn_1026} \\ 
&&\sqrt{1 + q} \simeq \frac{1 + 0.75 q}{1 + 0.25 q}, \hskip 3mm {\rm Claerbout\cite{Q_Claerbout}}
\label{eqn_1027} \\ 
&&\sqrt{1 + q} \simeq \frac{0.99987 + 0.79624 q}{1 + 0.30102 q}, \hskip 3mm {\rm Greene\cite{Q_Greene}}
\label{eqn_1028}
\end{eqnarray}
A a generalized PE applicable over a very wide angle is based on a Pade's series\cite{Q_Pade} 
\begin{eqnarray}
&&\sqrt{1 + q} = 1 + \sum_{s = 1}^m \frac{a_{sm}q}{1 + b_{sm}q} + O(q^{2m + 1})
\label{eqn_1029}
\end{eqnarray}
where 
\begin{eqnarray}
&&a_{sm} = \frac{2}{2m + 1} \sin^2 \left( \frac{s \pi}{2m + 1} \right),
\hspace{0.3in}
b_{sm} = \cos^2 \left( \frac{s \pi}{2m + 1} \right)
\nonumber
\end{eqnarray}
By substituting (\ref{eqn_1029}) into (\ref{eqn_1016}), a parabolic equation valid for a very wide angle is derived as
\begin{eqnarray}
&&\frac{\partial \psi}{\partial r} = i k_0 \sum_{\ell = 1}^m 
\frac{\displaystyle a_{\ell m}\left( n^2 - 1 + \frac{1}{k_0^2}\frac{\partial^2}{\partial z^2} \right)}
{\displaystyle 1 + b_{\ell m}\left( n^2 - 1 + \frac{1}{k_0^2}\frac{\partial^2}{\partial z^2} \right)} \psi
\end{eqnarray}
Wide-angle ($ > 20^\circ$) PE approximations require numerical methods like finite-difference or finite-element techniques.

\section{Reflection and Refraction}

\begin{figure}[h]
\vskip 5 cm
\hskip 0.5 cm
\special{wmf: Reflection_and_Refraction.jpg x=6 cm y=4.5 cm}
\caption{wave reflection and transmission at a interface.}
\label{Fig_pic22}
\end{figure}
Fig.(\ref{Fig_pic22}) shows the wave on the boundary\cite{COA}, where the pressure and vertical particle velocity should be continuous at the interface, namely,
\begin{eqnarray}
&&p_i + p_r = p_t
\label{RR_01a}\\
&&\frac{\partial p_i /\partial z}{\rho_1} + \frac{\partial p_r /\partial z}{\rho_1} 
= \frac{\partial p_t /\partial z}{\rho_t}  
\label{RR_01b}
\end{eqnarray}
We choose $c_1$ as reference sound speed. By using (\ref{eqn_0009}), we obtain
\begin{eqnarray}
&&p_i(r, z) = \psi_i(r, z) \sqrt{\frac{2}{\pi k_0 (r - r_0)}} e^{i (k_0 (r - r_0) - \pi / 4)}
\label{RR_02} \\
&&p_r(r, z) = \psi_r(r, z) \sqrt{\frac{2}{\pi k_0 (r - r_0)}} e^{i (k_0 (r - r_0) - \pi / 4)}
\label{RR_03} \\
&&p_t(r, z) = \psi_t(r, z) \sqrt{\frac{2}{\pi k_0 (r - r_0)}} e^{i (k_0 (r - r_0) - \pi / 4)}
\label{RR_04}
\end{eqnarray}
where $p_i$, $p_r$ and $p_t$ are pressure associated with incident, refleted and transmitted waves, respectively; $k_1 = 2 \pi f / c_1 $ and $k_2 = 2 \pi f / c_2$ are wavenumbers.
By referring to (\ref{eqn_1016.2}), $p_i$, $p_r$ and $p_t$ can be represented as
\begin{eqnarray}
&&\psi_i (r, z) = \psi_0 (z) e^{i k_0 (r - r_0) (Q_i - 1)} 
\label{RR_05}\\
&&\psi_r (r, z) = \psi_0 (z) R e^{i k_0 (r - r_0) (Q_r - 1)} 
\label{RR_06}\\
&&\psi_t (r, z) = \psi_0 (z) T e^{i k_0 (r - r_0) (Q_t - 1)} 
\label{RR_07}
\end{eqnarray}
where $R$ and $T$ are reflection and transmission coefficients, respectively.
Eqn.(\ref{RR_01a}) can then be reduced to
\begin{eqnarray}
&&e^{i k_0 Q_i (r - r_0)} + R e^{i k_0 Q_r (r - r_0)} = T e^{i k_0 Q_t (r - r_0)}
\label{RR_08}
\end{eqnarray} 
where
\begin{eqnarray}
&&Q_i = \sqrt{1 + q_i} = \sqrt{1 + n_1^2 - 1 - n_1^2 \sin^2 \theta_i} = n_1 \cos \theta_i  
\label{RR_09}\\
&&Q_r = \sqrt{1 + n_1^2 - 1 - n_1^2 \sin^2 \theta_r} = n_1 \cos \theta_r
\label{RR_10}\\
&&Q_t = \sqrt{1 + n_2^2  - 1 - n_2^2 \sin^2 \theta_t} = n_2 \cos \theta_t
\label{RR_11}
\end{eqnarray}
By the law of reflection, $\theta_r = - \theta_i$, we have $Q_r = Q_i$.
By the Snell's law, $n_2 \cos \theta_t = n_1 \cos \theta_i$, we have $Q_t = Q_i$.
Then, (\ref{RR_08}) implies that 
\begin{eqnarray}
&&1 + R = T 
\label{RR_12}
\end{eqnarray}

By substituting (\ref{RR_02}), (\ref{RR_03}) and (\ref{RR_04}) into (\ref{RR_01b}) and omitting the common terms, we obtain
\begin{eqnarray}
&&\frac{1}{\rho_1}\frac{\partial (\psi_i + \psi_r)}{\partial z}
= \frac{1}{\rho_2}\frac{\partial \psi_t}{\partial z}   
\label{RR_101}
\end{eqnarray}
From (\ref{RR_05})-(\ref{RR_07}), we have
\begin{eqnarray}
&&\frac{\partial \psi_\alpha (r, z)}{\partial z}
= \left[ \frac{\partial \psi_0 (z)}{\partial z} + i k_0 (r - r_0) \frac{\partial Q_\alpha(z)}{\partial z} \right] 
\exp \left\{ i k_0 (r - r_0) (Q_\alpha - 1) \right\} 
\nonumber \\
&&\simeq i k_0 (r - r_0) \frac{\partial Q_\alpha(z)}{\partial z} \exp \left\{ i k_0 (r - r_0) (Q_\alpha - 1) \right\} 
\label{RR_13}
\end{eqnarray}
with $\alpha = i, r, t$.
By substituting (\ref{RR_13}) into (\ref{RR_101}) and omitting the common expential and coefficient terms, we obtain
\begin{eqnarray}
&&\frac{1}{\rho_1}\left(\frac{\partial Q_i}{\partial z}+ R\frac{\partial Q_r}{\partial z}\right) = T\frac{1}{\rho_2}\frac{\partial Q_t}{\partial z} 
\label{RR_102}
\end{eqnarray}
Since $z = \sqrt{(r - r_0)^2 + z^2} \sin \theta$, we have
$\displaystyle \frac{\partial z}{\partial \theta} = \sqrt{(r - r_0)^2 + z^2} \cos \theta$.
Then, from (\ref{RR_09}), (\ref{RR_10}) and (\ref{RR_11}),
\begin{eqnarray}
&&\frac{\partial Q_i}{\partial z} = \frac{\partial Q_i}{\partial \theta_i}\frac{\partial \theta_i}{\partial z}
= -\frac{n_1 \sin \theta_i}{\sqrt{(r - r_0)^2 + z^2} \cos \theta_i} = - \frac{n_1 \tan \theta_i}{\sqrt{(r - r_0)^2 + z^2}}
\label{RR_14}\\
&&\frac{\partial Q_r}{\partial z} = -\frac{n_1 \sin \theta_r}{\sqrt{(r - r_0)^2 + z^2} \cos \theta_r} 
= - \frac{n_1 \tan \theta_r}{\sqrt{(r - r_0)^2 + z^2}}
= -\frac{\partial Q_i}{\partial z} \hspace{0.3cm} 
\label{RR_15}\\
&&\frac{\partial Q_t}{\partial z} = -\frac{n_2 \sin \theta_t}{\sqrt{(r - r_0)^2 + z^2} \cos \theta_t} 
= - \frac{n_2 \tan \theta_t}{\sqrt{(r - r_0)^2 + z^2}}
\label{RR_16} 
\end{eqnarray}
By subtituting (\ref{RR_14})-(\ref{RR_16}) into (\ref{RR_102}), we obtain
\begin{eqnarray}
&&\frac{(1 - R)}{\rho_1}\frac{n_1 \tan \theta_i}{\sqrt{(r - r_0)^2 + z^2}}
= \frac{T}{\rho_2}\frac{n_2 \tan \theta_t}{\sqrt{(r - r_0)^2 + z^2}}
\nonumber
\end{eqnarray}
or
\begin{eqnarray}
&&1 - R = T \frac{\rho_1}{\rho_2} \frac{n_2}{n_1} \frac{\tan \theta_t}{\tan \theta_i} 
= \frac{\rho_1 c_1 / \tan \theta_i}{\rho_2 c_2 / \tan \theta_t} T
\label{RR_103} 
\end{eqnarray}
since $n_1 / n_2 = c_2 / c_1$.
By solving (\ref{RR_12}) and (\ref{RR_103}), we obtain
\begin{eqnarray}
&&R = \frac{Z_2 - Z_1}{Z_2 + Z_1}, \hspace{0.3in}
T = \frac{2 Z_2}{Z_2 + Z_1}
\label{RR_20}
\end{eqnarray}
where 
\begin{eqnarray}
&&Z_1 = \frac{\rho_1 c_1}{\tan \theta_i}, \hspace{0.3in}
Z_2 = \frac{\rho_2 c_2}{\tan \theta_t}
\label{RR_19}
\end{eqnarray}

\textcolor{blue}{
Scattering is a mechanism for loss, interference and fluctuation. 
A rough sea surface or seafloor causes attenuation of the mean acoustic field. 
The attenuation increases with increasing frequency.
The backscattered field, moving ocean surface can generate fluctuations,
, so does bottom roughness when the sound source or receiver is moving.
The effect of scattering from a rough surface is thought of an additional loss.
If the roughness is is small with respect to the acoustic wavelength and the boundary can be modeled as a randomly rough surface, the reflection loss can be considered to be modified in a simple fashion by the scattering process as
\begin{eqnarray}
&&R'(\theta_i) = R(\theta_i) e^{-0.5 \Gamma^2}
\label{RR_20}\\
&&\Gamma = 2 k \sigma \sin \theta_i
\label{RR_21}
\end{eqnarray}
Since $\theta_t$ is function of $\theta_i$, the reflect coefficient can be presented as $R(\theta_i)$. Where $\Gamma$ is the Rayleigh roughness parameter, $k$ is wave number and $\sigma$ is the rms roughness.
}

\section{Phase Errors and Angular Limitations}

The accuracy of a parabolic approximation can be quantified with phase error.
Assume that the refractive index of the medium, $n(z) = k(z) / k_0$, is a function of depth only.
By substituting (\ref{eqn_1017.1}) and (\ref{eqn_1027}) into (\ref{eqn_1016}), we obtain
\begin{eqnarray}
&&\frac{\partial \psi}{\partial r} = i k_0 \left(\frac{1 + 0.75q}{1 + 0.25q} - 1 \right) \psi 
\label{eqn_1029.1}
\end{eqnarray}
which can be rewretten as
\begin{eqnarray}
&&\left( 3k_0^2 + k^2(z) + \frac{\partial^2}{\partial z^2} \right) \frac{\partial \psi}{\partial r}
= 2i k_0 \left( k^2(z) - k_0^2 + \frac{\partial^2}{\partial z^2} \right) \psi
\label{eqn_1029.2}
\end{eqnarray}

Next, decompose $\psi$ into
\begin{eqnarray}
&&\psi = \Phi(r) \Psi(z)
\label{eqn_1029.3}
\end{eqnarray}
where $\Phi(r)$ and $\Psi(z)$ are functions of $r$ and $z$, respectively.
Thus, (\ref{eqn_1029.2}) can be rearranged as
\begin{eqnarray}
&&\left( \frac{d^2 \Psi}{d z^2} + k^2(z) \Psi \right) \left( \frac{d \Phi}{d r} - 2 i k_0 \Phi \right) 
+ \left( 3 k_0^2 \frac{d \Phi}{d r} + 2 i k_0^3 \Phi \right) \Psi = 0
\label{eqn_1029.4}
\end{eqnarray}
which can be solved by imposing
\begin{eqnarray}
&&\frac{d^2 \Psi}{d z^2} + k^2(z) \Psi  = k_{rm}^2 \Psi, \qquad 
3 k_0^2 \frac{d \Phi}{d r} + 2 i k_0^3 \Phi  = - k_{rm}^2 \left( \frac{d \Phi}{d r} - 2 i k_0 \Phi \right)
\label{eqn_1029.4a}
\end{eqnarray}
where $k_{rm}^2$ is a separation constant. 
Eqn.(\ref{eqn_1029.4a}) is reorganized as
\begin{eqnarray}
&&\frac{d^2 \Psi}{d z^2} + \left( k^2(z) - k_{rm}^2 \right) \Psi = 0
\label{eqn_1029.4}\\
&&\frac{d \Phi}{d r} = i k_0 \left( \frac{2 k_{rm}^2 - 2 k_0^2}{3 k_0^2 + k_{rm}^2} \right) \Phi
\label{eqn_1029.5}
\end{eqnarray}
The solution to (\ref{eqn_1029.5}) is
\begin{eqnarray}
&&\Phi(r) = \Phi(r_0) \exp \left\{ i k_0 \left(\frac{2k_{rm}^2 - 2k_0^2}{3k_0^2 + k_{rm}^2} \right) (r - r_0) \right\}
\label{eqn_1029.6}
\end{eqnarray}
where $r_0$ is the source position. 
By substituting (\ref{eqn_1029.6}) into (\ref{eqn_0006}), we have
\begin{eqnarray}
&&p(r, z) = \Psi(z) \Phi(r_0) 
\exp \left\{ i k_0 \left(\frac{2k_{rm}^2 - 2k_0^2}{3k_0^2 + k_{rm}^2} \right) (r - r_0) \right\} H_0^{(1)}(k_0 r)
\nonumber\\
&&= \frac{p(r_0, z) H_0^{(1)}(k_0 r)}{H_0^{(1)}(k_0 r_0)}
\exp \left\{ i k_0 \left(\frac{2k_{rm}^2 - 2k_0^2}{3k_0^2 + k_{rm}^2} \right) (r - r_0)\right\}
\nonumber\\
&&\simeq p(r_0, z) \sqrt{r_0 / r}
\exp \left\{ i k_0 \left(\frac{3k_{rm}^2 + k_0^2}{3k_0^2 + k_{rm}^2}\right) (r - r_0) \right\}
\label{eqn_1029.7}
\end{eqnarray}
where (\ref{eqn_0008}) is applied in the last step.

In (\ref{eqn_1029.7}), define $k_{rm} = k_0 \cos \theta_m$ and set $k_0 = 1$ to simplify the modal phase as
\begin{eqnarray}
&&\varphi = \frac{1 + 3 \cos^2 \theta}{3 + \cos^2 \theta}
= \frac{4 - 3 \sin^2 \theta}{4 - \sin^2 \theta}
= \frac{1 - 0.75 \sin^2 \theta}{1 - 0.25 \sin^2 \theta} \qquad {\rm Claerbout}
\label{eqn_1029.9}
\end{eqnarray}
which is the same as that by substituting (\ref{eqn_1022}) into (\ref{eqn_1027}).

By the definition of $Q$ operator in (\ref{eqn_1017.1}), variable approximations of $\sqrt{1 + q}$ in (\ref{eqn_1026}), (\ref{eqn_1027}), 
(\ref{eqn_1028}) and (\ref{eqn_1029}) can be rewritten as
\begin{eqnarray}
{\rm Square-root} \hspace{0.5cm}		
&&Q = \sqrt{1 - \sin^2\theta}
\label{eqn_1029.10}\\
{\rm Tappert}\hspace{0.5cm}		
&&Q_1 = 1 - \frac{\sin^2 \theta}{2}
\label{eqn_1029.11}\\
{\rm Claerbout, Pade \ (1)}\hspace{0.5cm}	
&&Q_2 = \frac{1 - 0.75 \sin^2\theta}{1 - 0.25 \sin^2\theta}
\label{eqn_1029.12}\\
{\rm Green}\hspace{0.5cm}		
&&Q_3 = \frac{0.99987 - 0.79624 \sin^2\theta}{1 - 0.30102 \sin^2\theta}
\label{eqn_1029.13}\\
{\rm Pade \ (2)}\hspace{0.5cm}		
&&Q_4 = 1 - \frac{0.13820 \sin^2\theta}{1 - 0.65451 \sin^2\theta} - \frac{0.36180 \sin^2\theta}{1 - 0.09549 \sin^2\theta} 
\label{eqn_1029.14}\\
{\rm Pade \ (5)}\hspace{0.5cm}		
&&Q_5 = \sqrt{1 + q} = 1+ \sum_{s = 1}^{5} \frac{a_{sm}q}{1 + b_{sm}q}
\label{eqn_1029.15}
\end{eqnarray}
where 
\begin{eqnarray}
&&m = 5, \hspace{0.3cm} a_{sm} = \frac{2}{2 m + 1}\sin^2 \left(\frac{a \pi}{2m + 1} \right)
,\hspace{0.3cm} b = \cos^2 \left(\frac{a \pi}{2m + 1} \right)
\end{eqnarray}
Fig.\ref{Fig_pic1} and Fig.\ref{Fig_pic2} shows the phase errors $|Q_\alpha - Q |$ of various PE approximations $Q_\alpha$ as compared with the square-root $Q$. Fig.\ref{Fig_pic1}(a) is the simulation result and (b) is reference solution\cite{COA}, the same as Fig.\ref{Fig_pic2}.

\begin{figure}[h]
\vskip 4.5 cm
\hskip 1 cm
\special{wmf: COA_6_1_a_delta_th=1.jpg x=6.5cm y=5cm}
\hskip 7 cm
\special{wmf: COA_6_1_a_ref.jpg x=6.5cm y=5cm}
\vskip 0 cm 
\hskip 4 cm (a) \hskip 7 cm (b) 
\caption{Phase errors of various PE approximations at different propagation angles for reference error is 0.002.}
\label{Fig_pic1}
\end{figure}

\begin{figure}[h]
\vskip 5.5 cm
\hskip 1 cm
\special{wmf: COA_6_1_b_delta_th=0.1.jpg x=6.5cm y=5cm}
\hskip 7 cm
\special{wmf: COA_6_1_b_ref.jpg x=6.5cm y=5cm}
\vskip 0 cm 
\hskip 4 cm (a) \hskip 7 cm (b) 
\caption{Phase errors of various PE approximations at different propagation angles for reference error is 0.0002.}
\label{Fig_pic2}
\end{figure}
\textcolor{blue}{
\section{The Split-Step Fourier Algorithm}
Define the complex Fourier transform pair\cite{HT_App_SSF}\cite{AESD},
\begin{eqnarray}
&&\psi(r, z) = \frac{1}{2 \pi} \int^\infty_{-\infty} \tilde{\psi} (r, K_z) e^{i K_z z} d K_z
\label{eqn_1051}, \\
&&\tilde{\psi} (r, K_z) = \int^\infty_{-\infty} \psi(r, z) e^{-i K_z z} dz
\label{eqn_1052},
\end{eqnarray}
and using the integral skill,
\begin{eqnarray}
&&\int^\infty_{-\infty}\frac{\partial^2 \psi(r, z)}{\partial z^2} e^{-i K_z z} dz = - K_z^2 \tilde{\psi} (r, K_z)
\label{eqn_1054}
\end{eqnarray}
The Fourier transform of (\ref{eqn_1024}) is
\begin{eqnarray}
&&2 i k_0 \frac{\partial \tilde{\psi}}{\partial r} - K_z^2 \tilde{\psi} + k_0^2 (n^2 - 1)\tilde{\psi} = 0
\end{eqnarray}
or 
\begin{eqnarray}
&&\frac{\partial \tilde{\psi}}{\partial r} + \frac{k_0^2 (n^2 - 1) - K_z^2}{2 i k_0}\tilde{\psi} = 0
\label{eqn_1071}
\end{eqnarray}
which can be integrated in $r$ to have
\begin{eqnarray}
&&\tilde{\psi}(r, K_z) = \tilde{\psi}(r_0, K_z) \exp\left[-\frac{k_0^2 (n^2 - 1) - K_z^2}{2 i k_0}(r - r_0)\right]
\label{eqn_1072}
\end{eqnarray}
By inverse transforming (\ref{eqn_1072}) to the $z$-domain, we have
\begin{eqnarray}
&&\psi(r, z) = e^{( i k_0 / 2 ) (n^2 - 1)(r - r_0)} 
\int^\infty_{-\infty} \tilde{\psi}(r_0, K_z) e^{- i (r - r_0) K_z^2 / (2 k_0)} e^{i K_z z} d K_z
\label{eqn_1073}
\end{eqnarray}
which can be represented as
\begin{eqnarray}
&&\psi(r, z) = e^{(ik_0 / 2) [ n^2 (r_0, z) - 1 ] \Delta r}
{\cal F}^{-1} \left\{ e^{- i \Delta r K_z^2 / (2 k_0)} 
{\cal F} \{\psi( r_0 , z) \} \right\}
\label{eqn_1074}
\end{eqnarray}
where $\Delta r = r - r_0$. 
}
\\
\textcolor{blue}{
To improve the accuracy of the numerical scheme, in terms of $\Delta r$, first rewrite (\ref{eqn_1012}) as
\begin{eqnarray}
&&\frac{\partial \psi}{\partial r} = (A + B) \psi = U(r, z)\psi
\label{eqn_1077}
\end{eqnarray}
with
\begin{eqnarray}
&&A = \frac{i k_0}{2}[ n^2(r, z) - 1]; \ B = \frac{i}{2 k_0} \frac{\partial^2}{\partial z^2}
\label{eqn_1076}
\end{eqnarray}
where $U = A + B$ is a composite operator, which depends on both $r$ and $z$.
The formal solution to (\ref{eqn_1077}) can be expressed as
\begin{eqnarray}
&&\psi(r, z) = \exp \left\{ \int_{r_0}^{r_0 + \Delta r} U(r, z) dr \right\} \psi(r_0, z)
\simeq e^{\tilde{U} \Delta r} \psi(r_0,z)
\label{eqn_1078}
\end{eqnarray}
where the behavior of $U(r, z)$ over the range interval $\Delta r$ is approximated by $\tilde{U}$. 
Next, split the exponential operator $\exp[(A + B) \Delta r]$ into four alternative forms as 
\begin{eqnarray}
&&{\rm (I)}: e^{(A + B) \Delta r} \simeq e^{A \Delta r} e^{B\Delta r}
\label{eqn_1079}\\
&&{\rm (II)}: e^{(A + B)\Delta r} \simeq e^{B\Delta r} e^{A\Delta r}
\label{eqn_1080}\\
&&{\rm (III)}: e^{(A + B)\Delta r} \simeq e^{A \Delta r / 2} e^{B \Delta r}e^{A \Delta r / 2}
\label{eqn_1081}\\
&&{\rm (IV)}: e^{(A + B)\Delta r} \simeq e^{B \Delta r / 2} e^{A \Delta r} e^{B \Delta r / 2}
\label{eqn_1082}
\end{eqnarray}
Take (\ref{eqn_1079}) for example, let
\begin{eqnarray}
&&V(r_0, z) = e^{B \Delta r}\psi(r_0, z), \ {\rm with} \  B = \frac{i}{2 k_0} \frac{\partial^2}{\partial z^2}
\label{eqn_1083}
\end{eqnarray}
which is expanded as
\begin{eqnarray}
&&V(r_0, z) = \left[1 + \Delta r B + \frac{(\Delta r)^2}{2} BB + \cdots \right] \psi(r_0, z)
\nonumber\\
&&= \left[ 1 + \frac{i \Delta r}{2 k_0} \frac{\partial^2}{\partial z^2}  
+ \frac{1}{2} \left( \frac{i \Delta r}{2 k_0} \right)^2 \frac{\partial^4}{\partial z^4} + \cdots \right] \psi(r_0, z)
\label{eqn_1084}
\end{eqnarray}
By using (\ref{eqn_1054}), (\ref{eqn_1084}) is reduced to
\begin{eqnarray}
&&\tilde{V}(r_0, K_z) = \left[1 - \frac{i \Delta r}{2 k_0} K_z^2  + \frac{1}{2} \left(\frac{i \Delta r}{2 k_0} \right)^2 K_z^4 
+ \cdots \right] \tilde{\psi}(r_0, K_z)
 = e^{- i K_z^2 \Delta r / (2 k_0)} \psi(r_0, K_z)
\label{eqn_1085}
\end{eqnarray}
which can be formally represented as
\begin{eqnarray}
&&V(r_0, z) = {\cal F}^{-1} \left\{ e^{- i K_z^2 \Delta r / (2 k_0)} {\cal F}\{ \psi(r_0, z) \} \right\}
\label{eqn_1087}
\end{eqnarray}
Only the Tappert equation can be solved by the split-step Fourier technique.
}

\textcolor{blue}{
Hence, the split-step marching solutions to each of the four splittings in (\ref{eqn_1079})-(\ref{eqn_1082}) can be represented as
\begin{eqnarray}
&&{\rm (I)}\qquad \psi_{\rm I} (r, z) = e^{( i k_0 / 2) [ n^2 (r_0, z) - 1 ] \Delta r}
{\cal F}^{-1} \left\{ e^{- i \Delta r K_z^2 / (2 k_0)} {\cal F} \left\{ \psi(r_0, z) \right\} \right\}
\label{eqn_1088} \\
&&{\rm(II)}\qquad \psi_{\rm II}(r, z) = {\cal F}^{-1} \left\{ e^{- i \Delta r K_z^2 / (2 k_0)} 
{\cal F} \left\{ e^{(i k_0 / 2) [ n^2(r_0, z) - 1] \Delta r} \psi(r_0, z) \right\} \right\}
\label{eqn_1089} \\
&&{\rm (III)}\qquad \psi_{\rm III}(r, z) = e^{(i k_0 / 4) [ n^2 (r_0, z) - 1] \Delta r}
{\cal F}^{-1} \left\{ e^{- i \Delta r K_z^2 / (2 k_0)} {\cal F} \left\{ e^{(i k_0 / 4) [n^2 (r_0, z) - 1] \Delta r} \psi(r_0, z) \right\} \right\}
\label{eqn_1090}\\
&&{\rm (IV)}\qquad \psi_{\rm IV}(r, z) = {\cal F}^{-1} \left\{ e^{- i \Delta r K_z^2 / (4 k_0)}
{\cal F} \left\{ e^{(i k_0 / 2) [ n^2(r_0, z) - 1] \Delta r} {\cal F}^{-1} \left\{ e^{- i \Delta r K_z^2 / (4 k_0)} 
{\cal F} \left\{ \psi(r_0, z) \right\} \right\} \right\} \right\}
\label{eqn_1091}
\end{eqnarray}
\begin{figure}[h]
\vskip 5.5 cm
\hskip 1 cm
\special{wmf: sea_surround.jpg x=6.5cm y=5cm}
\caption{Schematic of solution domain for parabolic wave equations.}
\label{Fig_surround}
\end{figure}
}
\subsection{process boundary condition, volume attenuation and variable density in shallow water}

A schematic of the solution domain is shown in Fig.(\ref{Fig_surround}). \\
Sea surface is pressure-release boundary.
We assume there is no reflected wave at $z = z_{max}$ to simulate a bottom continuation by a homogeneous halfspace.
Terminate the physical solution domain $(0 \leq z \leq H)$ by an artificial absorption layer of several wavelengths thickness.
The absorption layer $(H < z \leq z_{max})$ is modeled with a complex index of refraction of the form\cite{DWC_NORDA}
\begin{eqnarray}
&& n^2 = n^2_b + i \alpha \exp \left[ -\left (\frac{z - z_{max}}{D} \right )^2\right]
\label{eqn_1092}
\end{eqnarray}
where $n_b = c_0 / c_b$, and generally the constants $\alpha = 0.01$ and $D = (z_{max} - H) / 3$.
\textcolor{blue}{
The thickness of the absorption layer is taken to be $H/3$ in deep water. However, to ensure meaningful results in shallow water, it is prudent to include a real physical bottom of several wavelengths thickness and with real sediment attenuation. 
}
\\   
Volume attenuation is included in the PE by adding a small imaginary part in to the medium wavenumber,\cite{B_twe}
\begin{eqnarray}
&&k = \omega/c + i \alpha , \ \ \alpha>0
\end{eqnarray}
where $\alpha$ is attenuation coefficient with unit nepers/m.
However, it is common in underwater acoustics the quantity $\alpha^{(\lambda)}$ replaces $\alpha$,
\begin{eqnarray}
&&\alpha^{(\lambda)} = - 20 \log \left( \frac{e^{- \alpha(r + \lambda)}}{e^{- \alpha r}} \right) = \alpha \lambda 20 \log e
\end{eqnarray}
where the unit is dB/$\lambda$, $\lambda$ is the acoustic wavelength.
In the lossy media we need to introduce a complex index of refraction as
\begin{eqnarray}
&&n^2 = \left(\frac{k}{k_0} \right)^2 \simeq \left(\frac{c_0}{c} \right)^2 \left[1 + i \frac {2 \alpha c}{\omega} \right] = \left(\frac{c_0}{c} \right)^2 \left[1 + i \frac {\alpha^{(\lambda)}}{27.29} \right]
\label{eqn_1093}
\end{eqnarray}
\\
There is an discontinuous on water-bottom interface should be considered, as show in appendix. By solving (\ref{eqn_0001}) for a variable-density medium in the form
\begin{eqnarray}
&&\rho \nabla \cdot \left (\frac{1}{\rho} \nabla p \right) + k_0^2 n^2 p = 0
\label{eqn_1094}\\
&&\nabla P + k_0^2 \tilde{n}^2 \hat{p} = 0, \ \ \ P = \frac{p}{\sqrt{\rho}}
\label{eqn_1095} \\
&&\nabla \psi_{ad} + k_0^2 \tilde{n}^2 \psi_{ad} = 0, \ \psi_{ad} = \frac{\psi}{\sqrt{\rho}} = \frac{P}{\sqrt{r}}
\label{eqn_1096}
\end{eqnarray}
where subscript $ad$ mean the variable is adjosted.
(\ref{eqn_1096}) is standard Helmholtz form like (\ref{eqn_0003}), so we can use (\ref{eqn_1079})-(\ref{eqn_1082}).
Where the subscript 1 mean the parameters at interface.
Where $\tilde{n}$ is the effective index of refraction given by (\ref{A15}) in Appendix, 
\begin{eqnarray}
&&\tilde{n}^2 = n^2 + \frac{1}{2k_0^2}\left[ \frac{1}{\rho} \frac{\partial^2 \rho}{\partial z^2} - \frac{3}{2 \rho^2} \left(\frac{\partial \rho}{\partial z} \right)^2 \right]
\label{eqn_1097}
\end{eqnarray}
It is clear that a density discontinuity with infinite derivatives causes problems in the
numerical algorithm. We, therefore, follow Tappert and introduce a smoothing
function at a given horizantal location of the form \cite{Tappert_WPU_tpam}(\cite{Tappert_WPU_tpam} library does not have this book)
\begin{eqnarray}
&&\rho(z) = \frac{1}{2}(\rho_b + \rho_0) + \frac{1}{2}(\rho_b - \rho_0) \tanh\left( \frac{z - z_b}{L} \right)
\label{eqn_1098}
\end{eqnarray}
where $z_b$ is the depth of the interface, $L$ is the distance over which the density
changes from $\rho_0$ to $\rho_b$ and must be small compared to the vertically projected wavelength.
An appropriate value of $L$ is given by
\begin{eqnarray}
&&k_0 L = 2
\label{eqn_1099}
\end{eqnarray}
Now we can solve acoustic signal by (\ref{eqn_1096}) and times $\sqrt{\rho(r,z)}$ to obtain the true result.

\subsection{Error Analysis and Grid Size Selection}
\textcolor{blue}{
We start with the parabolic wave equation (\ref{eqn_1077}) and assume $\psi_\ell = \psi(r_\ell, z)$is known at range $r_\ell$.\cite{DWC_NORDA} 
We use Taylor series expansion to find the field at $r_{\ell + 1} = r_\ell + \Delta r$
\begin{eqnarray}
&&\psi_{\ell + 1} = \psi_\ell + \psi_\ell ' \Delta r + \psi_\ell '' \frac{\Delta r ^ 2}{2}
+ \psi_\ell ''' \frac{\Delta r ^ 3}{6} + \cdots
\label{eqn_1100} \\
&&\psi_{\ell + 1} = \left[ 1 + U \Delta r + (U' + U^2) \frac{\Delta r ^ 2}{2}
+ (U'' + 2 U U' + U' U + U^3) \frac{\Delta r ^ 3}{6} \right]_\ell \psi_\ell
\label{eqn_1101}
\end{eqnarray}
where we now use primes to indicate differentiation with respect to range.
This solution, which is accurate to third order in $\Delta r ^ 3$, is used as a reference solution
Now we evaluate $U$ at range $r_\ell$ to find inherent error of (\ref{eqn_1078}) by expanding the exponential function to third order in $\Delta r$ to obtain
\begin{eqnarray}
 \psi_{\ell + 1} = \left[ 1 + U \Delta r + U^2 \frac{\Delta r ^ 2}{2}
+ U^3 \frac{\Delta r ^ 3}{6} \right]_\ell \psi_\ell
\label{eqn_1102}
\end{eqnarray}
The error is found as the difference between (\ref{eqn_1101}) and (\ref{eqn_1102})
\begin{eqnarray}
&&E_1 =  \frac{\Delta r ^ 2}{2} U_\ell ' \psi_\ell + O(\Delta r ^ 3)\psi_\ell
\label{eqn_1103} \\
&&U' = A' = \frac{i k_0}{2} \frac{\partial n^2}{\partial r}
\end{eqnarray}
Let us now turn to the second type of error caused by the operator splittings given
in (\ref{eqn_1079})-(\ref{eqn_1082}). 
We first consider a simple splitting of the form in (\ref{eqn_1079}), the error is again found by expanding the exponentials to third order in $\Delta r$ and comparing terms on both sides of the equality sign
\begin{eqnarray}
&&E_2 = -\frac{\Delta r ^ 2}{2} [AB - BA]\psi_\ell 
\nonumber \\
&&E_2 = -\frac{\Delta r ^ 2}{8} \left(\frac{\partial^2 n_\ell^2}{\partial z^2} \psi_\ell - 2\frac{\partial n_\ell^2}{\partial z} \frac{\partial \psi_\ell}{\partial z} \right)
\label{eqn_1104}
\end{eqnarray}
Combine (\ref{eqn_1103}) and (\ref{eqn_1104}), we obtain the value of $\Delta r$ affect the error.
For numerical analysis problem, we use finite-different method to find the first and second order partial differential equations.
\begin{eqnarray}
&&\frac{\partial n_\ell^2}{\partial r} = \frac{n_{\ell + 1} - n_\ell}{\Delta r}
\label{eqn_1104.1}
\end{eqnarray}
and for vertical partial differential,
\begin{eqnarray}
&&\frac{\partial n_\ell^2(z)}{\partial z} = \frac{n_\ell(z + \Delta z) - n_\ell(z)}{\Delta z}
\label{eqn_1104.2}\\
&&\frac{\partial^2 n_\ell^2(z)}{\partial z^2} = \frac{n_\ell^2(z - \Delta z) - 2 n_\ell^2(z) + n_\ell^2(z + \Delta z)}{\Delta z^2}
\label{eqn_1104.3}
\end{eqnarray}
The selection of $\Delta r$ depends on the acceptable error in transmission loss at the maximum propagation range considered, generally 1 dB or 3 dB.
}
The choice of $\Delta z$ depends on the required depth sampling of environmental discontinuities.
The vertical extent $z_l$ of the standard Gaussian source defined as the distance between the two $1/e$-decay points\cite{COA}
\begin{eqnarray}
&&\psi(0, z) = A_0 e^{-k_0^2(z - z_s)^2 / 2}
\nonumber \\
&&z_l = 2 \times \frac{\sqrt{2}}{k_0} \sim  \lambda / 2
\label{eqn_1105}
\end{eqnarray}
where $A_0$ is reference constanst and $z_s$ is the depth of source, . 
Hence, the depth sampling must be less than half a wavelength, and an adequate upper bound is given by $\Delta z \leq \lambda / 4$.


\section{Transmission Loss}
\textcolor{blue}{
An acoustic signal traveling through the ocean becomes distorted due to multipath
effects and weakened due to various loss mechanisms. The standard measure in
underwater acoustics of the change in signal strength with range is transmission
loss defined as the ratio in decibels between the acoustic intensity $I(r, z)$ at a field
point and the intensity $I_0$ at $1$-m distance from the source,\cite{COA}
\begin{eqnarray}
&&{\rm TL} (r, z) = - 10 \log_{10} \frac{I(r, z)}{I_0} \ {\rm (dB)}
= - 20 \log_{10} \frac{\left| p(r, z) \right|}{\left| p_0 \right|} \ {\rm (dB)}
\label{TL_0}
\end{eqnarray}
where $p_0$ is pressure at 1-m distance from the source.
Transmission loss may be considered to be the sum of a loss due to geometrical spreading and a loss due to attenuation.
Fig.(\ref{TL_pic1}) shows the two geometries of importance in underwater acoustics.
\begin{figure}[h]
\vskip 5 cm
\hskip 1 cm
\special{wmf: Spherical_spreading.jpg x=4.5cm y=3cm}
\hskip 6 cm
\special{wmf: Cylindrical_spreading.jpg x=4.5cm y=3cm}
(b)
\caption{(a) Spherical spreading and (b) Cylindrical spreading}
\label{TL_pic1}
\end{figure}
In Fig.(\ref{TL_pic1}.a), if we assume the medium to be lossless, the propatation energy is inversely proportional to the surface of the sphere.
\begin{eqnarray}
&&I(r, z) \propto \frac{1}{4 \pi R^2}, \ \ R = \sqrt{r ^ 2 + z ^ 2}
\nonumber\\
&&{\rm TL}(r, z) \propto 10 \log_{10} \frac{1}{R^2}
\label{TL_1}
\end{eqnarray}
where $A_0$ is a reference magnitude for
If the media has upper and down boundary like Fig.(\ref{TL_pic1}.b), the far-field intensity is inversely proportional to the surface of a cylinder of radius $r$ and depth $z_b$ at far-field ($r \gg z_b$). Where as
\begin{eqnarray}
&&I(r, z) \propto \frac{1}{2 \pi r z_b}
\nonumber\\
&&{\rm TL}(r, z) \propto 10 \log_{10} \frac{1}{r}
\label{TL_2}
\end{eqnarray}
Note that for a point source in a waveguide, we have spherical spreading in the
nearfield ($r \leq z_b$) followed by a transition region toward cylindrical spreading
which applies only at longer ranges ($r \gg z_b$).
}

\section{NUMERICAL EXAMPLE}
\subsection{EXAMPLE A}
\begin{figure}[h]
\vskip 5 cm
\hskip 1 cm
\special{wmf: Acoustic_wave_module_2.jpg x=6cm y=4.5cm}
\caption{The wave reflect by sea surface and bottom, and image source.}
\label{Fig_pic4}
\end{figure}
\begin{figure}[h]
\vskip 5 cm
\hskip 1 cm
\special{wmf: RSRB.jpg x=6cm y=4.5cm}
\caption{}
\label{Fig_pic7}
\end{figure}
\textcolor{blue}{
For the acoustics wave propagating in shallow water (depth less 200m), the wave will be reflectet by sea surface and bottom many times, showed in Fig.(\ref{Fig_pic4}). 
There is a acoustic signal source at $(r, z_s)$, let sea surface is $r$-axis and depth of bottom is $z_b$, where $z$ direction is positive for down.
To keep the pressure is release at sea surface ($\psi(r, z) = 0$ at $z = 0$), we add an image source at $(0, -zs)$ with the same amplitude to balance the wave from source to sea surface.
}


\begin{figure}[h]
\vskip 5 cm
\hskip 1 cm
\special{wmf: COA_6_2_10th_order.jpg x=6.5cm y=5cm}
\hskip 7 cm
\special{wmf: COA_6_2_ref.jpg x=6.5cm y=5cm}
\vskip 0cm
\hskip 4.5cm (a) \hskip 6cm (b)
\caption{Transmition loss of Tappert (red) and Claerbout (blue) approximations, (a) order 10, (b)  reference result and 
; in reference result, standard PE is Tappert PE, wide-angle PE is Claerbout PE;
$f = 250$ Hz, $z_b = 100$ m, $z_s = 99.5$ m, $z_r = 99.5$ m, 
$\rho_1 = 1.293 \times 10^{-3}$ g/cm$^3$, $c_1 = 340$ m/s, $\rho_0 = \rho_2 = 1$ g/cm$^3$, $c_0 = c_2 = 1,500$ m/s, 
$\rho_3 = \rho_b =  1.2$ g/cm$^3$, $c_3 = c_b = 1,590$ m/s, $\Delta r = 50$ m and $\Delta z = 1$ m.}
\label{Fig_pic3}
\end{figure}
\textcolor{blue}{
Fig.(\ref{Fig_pic3}.b) shows an example in \cite{COA} fig.6.2 about transmission loss with different PE approximations, Tappert's (Standard PE), Clearbout's (Wide-angle PE). 
The reference sound speed $c_0 = c_1$ is $1500$ m/s, and frequency $f$ is $250$ Hz.
Eqn.(\ref{TL_0}) at far-field can be rewriten  as
\begin{eqnarray}
&&{\rm TL}_f (r, z) = - 10 \log_{10} \frac{\left| p(r, z) \right|^2}{\left| p(r_0, z) \right|^2} 
= -10 \log_{10} \frac{\left| \psi(r, z) \right|^2 / r}{\left| \psi(r_0, z) \right|^2 / r_0}
\ {\rm (dB)} \ {\rm for \ }r \gg z_b
\label{ex_0}
\end{eqnarray}
where $r_0 = 1$ m 
}
If we only consider five kinds wave $\psi_d$, $\psi_b$, $\psi_{bs}$, $\psi_s$ and $\psi_{sb}$. By implying (\ref{eqn_1016.2}), the wave can be represented as
\begin{eqnarray}
&&\psi(r, z) = \psi_d (r, z) + \psi_s (r, z) + \psi_b (r, z) + \psi_{sb} (r, z) + \psi_{bs} (r, z) 
\label{ex_1}
\end{eqnarray}
where
\begin{eqnarray}
&&\psi_d(r, z) = \psi_0(z) e^{i k_0 r (Q_d - 1)}
\label{exa_2}\\
&&\psi_b(r, z) = \psi_0(z) R_b e^{i k_0 r (Q_b - 1)}
\label{exa_3}\\
&&\psi_s(r, z) = \psi_0(z) R_s e^{i k_0 r (Q_s - 1)}
\label{exa_4}\\
&&\psi_{bs}(r, z) = \psi_0(z) R_b R_s e^{i k_0 r (Q_{bs} - 1)}
\label{exa_5}\\
&&\psi_{sb}(r, z) = \psi_0(z) R_s R_b e^{i k_0 r (Q_{sb} - 1)}
\label{exa_6}
\end{eqnarray}
where $R_b$ and $R_s$ are the reflection coefficients at bottom and sea surface, respectively.
\textcolor{red}{Draw a figure to show $Z_1$, $Z_2$, $Z_3$, $\theta_1$, $\theta_2$, $\theta_3$.}
From (\ref{RR_19}),
\begin{eqnarray}
&&R_b = \frac{Z_3 - Z_2}{Z_3 + Z_2}, \ \  R_s = \frac{Z_1 - Z_2}{Z_1 + Z_2}
\\
&&Z_\alpha = \frac{\rho_\alpha c_\alpha}{\tan \left|\theta_\alpha \right|}, \hspace{0.3 cm} \alpha = 1, 2, 3
\end{eqnarray}
where $\theta_2$ is propagation angle at source, $\theta_1$ and $\theta_3$ are refracted angle in Region I and Region II, respectly. 
The sound speeds in air, water and seafloor re $c_1 = 340$ m/s, $c_2 = 1500$ m/s and $c_3 = 1590$ m/s, respectively;
$\rho_1 = 1.239 \times 10^{-3}$ g/cm$^3$, $\rho_2 = 1$ g/cm$^3$ and $\rho_3 = 1.2$ g/cm$^3$ are mass densities 
in air, water and seafloor, respectly.
Consider a Gaussian type of start field, 
\begin{eqnarray}
&&\psi_0(z) = A_0 e^{- k_0^2 (z - z_s)^2 / 2}
\label{exa_7}
\end{eqnarray}

(\ref{eqn_1029.11}),(\ref{eqn_1029.12}) and (\ref{eqn_1029.15}) show corresponding $Q$ for Tappert's, Claerbout's and Pade's approximations. 
Here we list propagating angles for each wave.
\begin{eqnarray}
&&\theta_d = \tan^{-1} \frac{z - z_s}{r}, 
\\
&&\theta_s = \tan^{-1} \frac{-(z + z_s)}{r},
\hspace{0.3cm} \theta_b = \tan^{-1} \frac{2 z_b - z - z_s}{r}
\\
&&\theta_{bs} = \tan^{-1} \frac{2 z_b - z_s + z_r}{r}, 
\hspace{0.3cm} \theta_{sb} = -\tan^{-1} \frac{2 z_b + z_s - z_r}{r}
\end{eqnarray}
The simulation result show in Fig.(\ref{Fig_pic3}.a) for (\ref{ex_1}), and reference result is Fig.(\ref{Fig_pic3}.c).\\
Now we increase reflection times and obtain
\begin{eqnarray}
&&\psi_{s(bs)n} = \psi_0(z) R_b^n R_s^{n + 1} e^{i k_0 r (Q_{s(bs)n} - 1)}
\\
&&\psi_{b(sb)n} = \psi_0(z) R_b^{n + 1} R_s^n \exp(i k_0 r (Q_{b(sb)^n} - 1))
\\
&&\psi_{s(bs)n b} = \psi_0(z) R_b^{n + 1} R_s^{n + 1} \exp(i k_0 r (Q_{s(bs)^n b} - 1))
\\
&&\psi_{b(sb)n s} = \psi_0(z) R_b^{n + 1} R_s^{n + 1} \exp(i k_0 r (Q_{b(sb)^n s} - 1))
\end{eqnarray}
where $n$ is the number of pair reflecting on sea surface and bottom. The corresponding propagation angles are
\begin{eqnarray}
&&\theta_{s(bs)^n} = -\tan^{-1}\frac{z_r + z_s + 2 n z_b}{r}, 
\\
&&\theta_{b(sb)^n} = \tan^{-1}\frac{(z_b - z_r) +(z_b - z_s) + 2 n z_b}{r} 
= \tan^{-1}\frac{-z_r - z_s + 2 (n + 1) z_b}{r}
\\
&&\theta_{s(bs)^n b} = -\tan^{-1}\frac{(z_b - z_r) + z_s + (2 n + 1) z_b}{r}
= -\tan^{-1}\frac{- z_r + z_s  + 2 (n + 1) z_b}{r}
\\
&&\theta_{b(sb)^n s} = \tan^{-1}\frac{z_r + (z_b - z_s) + (2 n + 1) z_b}{r} 
= \tan^{-1}\frac{z_r - z_s + 2 (n + 1) z_b}{r},
\end{eqnarray}

(\ref{ex_1}) can be rewriten as 
\begin{eqnarray}
&&\psi(r, z) = \psi_d + \sum_{n = 1}^{\infty} \left(\psi_{s(bs)^n} + \psi_{b(sb)^n} + \psi_{s(bs)^n} + \psi_{b(sb)^n s} \right)
\label{ex_10}
\end{eqnarray}
If $n$ increases, the propagation angles approach $90^\circ$, the reflection coefficienties would be very small so we can neglect them. 
The simulation result of (\ref{ex_10}) show in Fig.(\ref{Fig_pic3}.b).



\subsection{EXAMPLE B}
\begin{figure}[h]
\vskip 5 cm
\hskip 1 cm
\special{wmf: COA_6_3_TransLoss_99_5_FT_2.jpg x=6.5cm y=5.2cm}
\hskip 7 cm
\special{wmf: COA_6_3_radia_pattern_FT_2.jpg x=6.5cm y=5.2cm}
\vskip 0cm
\hskip 4.5cm (a) \hskip 6cm (b)
\vskip 5 cm
\hskip 1 cm
\special{wmf: COA_6_2_ref.jpg x=6.5cm y=5cm}
\hskip 4.5cm (c)
\caption{slip-step Fourier algorithm.(a)Fourier transform: transmition loss at $z = 99.5$ and (b) radiate pattern,(c) reference solution;
; $f = 250$ Hz, $z_b = 100$ m, $z_s = 99.5$ m, $z_r = 99.5$ m, 
%$\rho_a = 1.293 \times 10^{-3}$ g/cm$^3$, $c_a = 340$ m/s, 
$\rho_0 = 1$ g/cm$^3$, $c_0 = 1,500$ m/s, 
$\rho_b =  1.2$ g/cm$^3$, $c_b = 1,590$ m/s, $\Delta r = 0.5$ m and $\Delta z = 0.5$ m, 
$\alpha^{(\lambda)}_b = 0.5$ dB/$\lambda$, $H = 197$ m, $D = 19.67$ m, $L = 1.91$ m.}
\label{Fig_pic8}
\end{figure}
Here we use another way to solve the problem Fig. 6.2 in \cite{COA}, the slip-step Fourier algorithm. 
We apply (\ref{eqn_1088}) to find $\psi(r,z)$, and the simulation transmission loss show in Fig.(\ref{Fig_pic8}).
\textcolor{blue}{
We choose $3$dB as acceptable accuracy, which means 
\begin{eqnarray}
&& 20 \left| \log_{10} \left(\frac{\psi + E_1 + E_2}{\psi} \right) \right| \leq 3
\label{ex_11}
\end{eqnarray}
where $E_1$ and $E_2$ are defined as (\ref{eqn_1103}) and (\ref{eqn_1104}), and we can find the maximun of $\Delta r$.
(\ref{eqn_1105}) show the maximun of $\Delta z$.
}
The effective index of refraction $\tilde{n}$ definited from (\ref{eqn_1092}), (\ref{eqn_1095}) and (\ref{eqn_1097}) showed as follow 
\begin{eqnarray}
&&\tilde{n}^2(r, z) = \left\{ \begin{array}{ll}
n_0^2, \hspace{1in} & 0 < z \leq z_b - L / 2 \\ & \\
\displaystyle n_0^2 + \frac{1}{2k_0^2}\left[ \frac{1}{\rho} \frac{\partial^2 \rho}{\partial z^2} 
- \frac{3}{2 \rho^2} \left(\frac{\partial \rho}{\partial z} \right)^2 \right], &
z_b - L / 2 < z \leq z_b + L / 2 \\ & \\
\displaystyle n_b^2\left[1 + i \frac{\alpha^{(\lambda)}}{27.29} \right], &
z_b + L / 2 < z \leq H \\ & \\
\displaystyle n_b^2  \left[1 + i \frac{\alpha^{(\lambda)}}{27.29} \right] 
+ 0.01 \times i \exp \left[ -\left (\frac{z - z_{\max}}{D} \right )^2\right],
& H < z \leq z_{\max}
\end{array} \right. 
\label{EXB_1}
\end{eqnarray}
We use $\psi_{ad} = \psi/\sqrt{\rho}$ to replacing $\psi$ with the density as (\ref{eqn_1098}). 
As show before, we need to decide the grid distance for both calculation speed and accuracy. 
Applying (\ref{eqn_1103}) and (\ref{eqn_1104}), we know the error values depend on both the partial differential and grid size. 
With a limitation of error value, there is a maximun grid size.
In this problem, it is clear type one error ($E_1$) is zero, and only at the interface will be error two.
Then, we apply (\ref{eqn_1097}) with (\ref{eqn_1092_1}) and times the result with $\sqrt{\rho}$ to obtain the solution.
Here we list two different simulation way to compare, the resuls of Fourier transform show in Fig.(\ref{Fig_pic9}.(c)\&(d)).

\subsection{EXAMPLE C}
Here we solve the same problem in past two examples but extend computing domain to air.
In this example we apply (\ref{eqn_1088}) to find the solution.
At first exercise, we change the depth the of the absorption layer, the thickness equal to 1/3 from sea surface to begin of absorption layer. 
The results show in Fig.(\ref{Fig_pic11}). 
We can know the three line is almostly the same but near $7$ km at the range, so we just use the condition of red line to design the ABL under sea bottom for the second exercise.


%\begin{figure}[h]
%\vskip 5 cm
%\hskip 1 cm
%\special{wmf: sea_air_surround.jpg x=6.5cm y=5.2cm}
%\caption{Environment of EXAMPLE C}
%\label{Fig_pic13}
%\end{figure}

The second exercise is designing ABL in air. The environment show in Fig.(\ref{Fig_pic13}).
Like the list in (\ref{EXB_1}), we use (\ref{eqn_1092}), (\ref{eqn_1093}), (\ref{eqn_1097}) and (\ref{eqn_1098}) to find index of refraction in different depth and height, and list here. 
\textcolor{blue}{
By (\ref{EX_11}), if we choose $\Delta r = \Delta z = 0.5$m, the error is 1.318 dB at 7 km.  
The mass density approximation at interface is defined by (\ref{eqn_1098}) and so we can use (\ref{A16}) to find the index of refraction approximation at interface.
Here we show the first and second order partial differetial at bottom
\begin{eqnarray}
&&\rho(z) = \frac{\rho_b + \rho_0}{2} + \frac{\rho_b - \rho_0}{2} \tanh \left(\frac{z - z_b}{L_b}\right)
\\
&&\frac{\partial \rho(z)}{\partial z} = \frac{\rho_b - \rho_0}{2} \sech^2 \left(\frac{z - z_b}{L_b}\right)
\label{A17}\\
&&\frac{\partial^2 \rho(z)}{\partial z^2} = -\left(\rho_b - \rho_0 \right) \sech^2 \left(\frac{z - z_b}{L_b}\right) \tanh \left(\frac{z - z_b}{L_b}\right)
\label{A18}
\end{eqnarray}
where $k_0 L_b = 2$ is an appropriate value. 
Likely, the mass density approximation at air/water interface is
\begin{eqnarray}
&&\rho(z) = \frac{1}{2}(\rho_0 + \rho_a) + \frac{1}{2}(\rho_0 - \rho_a) \tanh\left(\frac{z - 0}{L_a} \right), \hspace{0.3mm}
\label{EXC_0}
\end{eqnarray}
But by the derivation in appendix, we should extend $L_a$ so the variable between each grid point is small enough.(Not yet find the way to dicide the value of $L_a$)
}
We use the information of \cite{a_a_t_h}, choose $0^\circ C$, 10 percent relative humidity, windless air and decide.
\begin{eqnarray}
&&\tilde{n}^2(r, z) = \left\{ \begin{array}{ll}
\displaystyle n_a^2 \left[1 + i \frac{\alpha_a^{(\lambda)}}{27.29} \right] + 0.01 \times i \exp \left[ -\left (\frac{z_{a\max} - z}{D_a} \right )^2\right], 
\hspace{1in} & -z_{a\max} < z \leq -H_a \\ & \\
%%%%%%%%%%%%%%%%%%%%%%%%%%%%%%%%%%%%
\displaystyle n_a^2 \left[1 + i \frac{\alpha_a^{(\lambda)}}{27.29} \right],
 \hspace{1in} &  -H_a < z \leq - L_a / 2 \\ & \\
%%%%%%%%%%%%%%%%%%%%%%%%%%%%%%%%%%%%%%%%%
\displaystyle n_a^2 + \frac{1}{2k_0^2}\left[ \frac{1}{\rho} \frac{\partial^2 \rho}{\partial z^2} 
- \frac{3}{2 \rho^2} \left(\frac{\partial \rho}{\partial z} \right)^2 \right], &
- L_a / 2 < z \leq L_a / 2 \\ & \\
%%%%%%%%%%%%%%%%%%%%%%%%%%%%%%%%%%%%%%%
n_0^2, \hspace{1in} & L_a / 2 < z \leq z_b - L_b / 2 \\ & \\
%%%%%%%%%%%%%%%%%%%%%%%%%%%%%%
\displaystyle n_b^2 + \frac{1}{2k_0^2}\left[ \frac{1}{\rho} \frac{\partial^2 \rho}{\partial z^2} 
- \frac{3}{2 \rho^2} \left(\frac{\partial \rho}{\partial z} \right)^2 \right], &
z_b - L_b / 2 < z \leq z_b + L_b / 2 \\ & \\
%%%%%%%%%%%%%%%%%%%%%%%%%%%%%%%%%%%
\displaystyle n_b^2\left[1 + i \frac{\alpha_b^{(\lambda)}}{27.29} \right], &
z_b + L_b / 2 < z \leq H_b \\ & \\
%%%%%%%%%%%%%%%%%%%%%%%%%%%%
\displaystyle n_b^2  \left[1 + i \frac{\alpha_b^{(\lambda)}}{27.29} \right] 
+ 0.01 \times i \exp \left[ -\left (\frac{z - z_{b\max}}{D_b} \right )^2\right],
& H_b < z \leq z_{b\max}
\end{array} \right. 
\label{EXC_1}
\end{eqnarray}
\textcolor{blue}{
Where $n_a$, $n_0$ and $n_b$ ares the indexes of refraction of air, water and bottom;
 $\alpha_a^{(\lambda)}$ and $\alpha_b^{(\lambda)}$ are attenuation coefficient in air\cite{a_a_t_h} and bottom; 
$z_{a\max}$ and $z_{a\max}$ are the up and down limitation of compute domain; $D_b = (H_b - z_{b\max}) / 3$ and  $D_a = (H_a - z_{a\max}) / 3$.}
\begin{figure}[h]
\vskip 5 cm
\hskip 1 cm
\special{wmf: ABL_sb_air_all.jpg x=6.5cm y=5.2cm}
\hskip 7 cm
\special{wmf: COA_6_2_ref.jpg x=6.5cm y=5.2cm} 
\vskip 0cm
\hskip 4.5cm (a) \hskip 6cm (b)
\caption{(a) TL with different thicknesses of ROI in the air,
(b) reference solution \textcolor{blue}{\cite{COA}}\textcolor{red}{\cite{??}};
$f = 250$ Hz, $z_s = 99.5$ m, $z_r = 99.5$ m, 
$\rho_0 = 1,000$ kg/m$^3$, $c_0 = 1,500$ m/s, 
$\rho_b = 1,500$ kg/m$^3$, $c_b = 1,590$ m/s, $\Delta r = 0.5$ m and $\Delta z = 0.5$ m, $L = 1.91$ m, 
$\alpha^{(\lambda)}_b = 0.5$ dB/$\lambda$,
temperature of air is 0$^\circ C$, relative humidity is 10 percent, sound speed in air is $c_a = 331.5 m/s$, 
attenuation of air $\alpha^{(\lambda)}_a = 0.0346$ dB/$\lambda$ \cite{a_a_t_h}.
the height of $H_a$ and $z_{a\max}$: both 0 m for red line, 1m and 2m for blue line, 5m and 10m for yellow line, 10m and 20m for green line and 50m and 100m for black line.}
\label{Fig_pic12}
\end{figure}

\section{Appendix}

Follow Bergmann \cite{B_twe}, here we derive wave equation with variable index of refraction.
Here list equation of continuity, motion and state,
\begin{eqnarray}
&&\frac{\partial \rho}{\partial t} + \nabla \cdot ( \rho \bar{u} ) = 0
\label{A1} \\
&&\rho \frac{\partial \bar{u}}{\partial t} + \rho ( \bar{u} \cdot \nabla ) \bar{u}
+ \nabla p + \rho \nabla V = 0
\label{A2} \\
&&\frac{\partial p}{\partial t} + \bar{u} \cdot \nabla p = \frac{\kappa}{\rho} \left( \frac{\partial \rho}{\partial t} + \bar{u} \cdot \nabla \rho \right)
\label{A3}
\end{eqnarray}
where $\rho$ is mass density of flow, $t$ is time, $\bar{u}$ is flow velocity and $V$ is potential of gravity, $\kappa$ is bulk modulus. 
Now we consider a static solution, in which the density $\rho_s$, the pressure $p_s$, and the bulk modulus $\kappa_s$ are independent of $t$, and the velocity vanishes ($\bar{u}_s = 0$), (\ref{A2}) reduces to 
\begin{eqnarray}
&&\nabla p_s + \rho_s \nabla V = 0
\label{A4}
\end{eqnarray}
and $\kappa_s$ is defined by means of virtual nonstatic variations of the solution.
Then we add small perturbation which subscript 1,
\begin{eqnarray}
&&\rho = \rho_s + \rho_1, \hspace{0.3 cm} \bar{u} = \bar{u}_s + \bar{u}_1 = \bar{u}_1 , 
\hspace{0.3 cm} p = p_s + p_1 , \hspace{0.3 cm} \kappa = \kappa_s + \kappa_1
\label{A4_1}
\end{eqnarray}
than we obtain (\ref{A1}), (\ref{A2}) and (\ref{A3}) become 
\begin{eqnarray}
&&\frac{\partial \rho_1}{\partial t} + \nabla \cdot [ (\rho_s + \rho_1) \bar{u}_1] = 0
\label{A5}\\
&&(\rho_s + \rho_1) \frac{\partial \bar{u}_1}{\partial t} + \nabla (p_s + p_1) + (\rho_s + \rho_1) \nabla V = 0
\label{A6}\\
&&\frac{\partial (p_s + p_1)}{\partial t} + \bar{u}_1 \cdot \nabla (p_s + p_1) = \frac{(\kappa_s + \kappa_1)}{(\rho_s + \rho_1)} \left( \frac{\partial \rho_1}{\partial t} + \bar{u}_1  \cdot \nabla (\rho_s + \rho_1) \right)
\label{A7}
\end{eqnarray}
\textcolor{blue}{Because the perturbation is small, we can ignore the product of two variable which subscript 1,(maybe don't used with the air layer.)}; and $\frac{(\kappa_s + \kappa_1)}{(\rho_s + \rho_1)} \sim \frac{\kappa_s}{\rho_s}$
 substituding (\ref{A5}) into (\ref{A7}) and we can obtain
\begin{eqnarray}
&&\frac{\partial p_1}{\partial t} + \bar{u}_1 \cdot \nabla p_s + \kappa_s \nabla \cdot \bar{u}_1 = 0 
\label{A8}
\end{eqnarray}
$p_1$ is the variable commonly referred to as the sound pressure, $\rho_1$ is the mass density change of the (compressible) fluid associated with the sound wave, and $\bar{u}_1$ is the instantaneous material velocity of the fluid.
Here we consider $p_s$ as constanst depent on $V$, and ignored the effect gravity, then (\ref{A6}) and (\ref{A8}) are simplified as
\begin{eqnarray}
&&\rho_s \frac{\partial \bar{u_1}}{\partial t} + \nabla p_1 = 0
\label{A9}\\
&&\frac{\partial p_1}{\partial t} + \kappa_s \nabla \cdot \bar{u_1} = 0
\label{A10}
\end{eqnarray}
(\ref{A9}) is divided by $\rho_s$, (\ref{A10}) is divide by $\kappa_s$ and differentiated
with respect to the time $t$, then we obtain
\begin{eqnarray}
&&\frac{1}{\kappa_s}\frac{\partial^2 p_1}{\partial t^2} - \nabla \cdot \frac{\nabla p_1}{\rho_s} = 0
\label{A11}\\
&&\nabla^2 p_1 - \frac{\rho_s}{\kappa_s}\frac{\partial^2 p_1}{\partial t^2} -\frac{1}{\rho_s} \nabla \rho_s \cdot \nabla p_1 = 0
\label{A12}
\end{eqnarray}
Introducing a new wave variable, namely
\begin{eqnarray}
&&P = \frac{p_1}{\sqrt{\rho_s}}
\label{A13}
\end{eqnarray}
Then (\ref{A12}) become
\begin{eqnarray}
&&\nabla^2 P - \frac{\rho_s}{\kappa_s}\frac{\partial^2 P}{\partial t^2} 
+ \left[\frac{\nabla^2 \rho_s}{2 \rho_s} - \frac{3}{4} \left( \frac{\nabla \rho_s}{\rho_s} \right)^2 \right] P = 0
\label{A14}
\end{eqnarray}
for time independent form of (\ref{A14}),
\begin{eqnarray}
&&\nabla^2 P + k_s n^2 P + \left[ \frac{\nabla^2 \rho_s}{2 \rho_s} - \frac{3}{4} \left(\frac{\nabla \rho_s}{\rho_s}\right)^2 \right] P = 0
\label{A15} \\
&&\nabla^2 P + k_s \tilde{n}^2 P = 0, \hspace{0.3cm} 
\tilde{n}^2 = n^2 + \left[ \frac{\nabla^2 \rho_s}{2 \rho_s} - \frac{3}{4} \left( \frac{\nabla \rho_s}{\rho_s} \right)^2 \right]
\label{A16}
\end{eqnarray}
where $k_s$ is the reference wavenumber.



\begin{thebibliography}{99}
\bibitem{COA}
F. B. Jensen, W. A. Kuperman, M. B. Porter, Henrik Schmidt,
Computational Ocean Acoustics, 2cd Edition.(1994)

\bibitem{wass}
D. J. Thomson, and N. R. Chapman
A wide-angle split-step algorithm for the parabolic equation

\bibitem{Tappert_WPU_tpam}
F.D. Tappert, The parabolic approximation method. in Wave Propagation in Underwater
Acoustics, ed. by J.B.Keller, J.S. Papadakis (Springer, New York, 1977), pp. 224�V287

\bibitem{HT_App_SSF}
R.H. Hardin, F.D. Tappert, Applications of the split-step Fourier method to the numerical solution
of nonlinear and variable coefficient wave equations

\bibitem{DWC_NORDA}
J.A. Davis, D. White, R.C. Cavanagh, NORDA parabolic equation workshop. Rep. TN-143.
Naval Ocean Research and Development Activity, Stennis Space Center

\bibitem{Q_Claerbout}
J.F. Claerbout, Fundamentals of Geophysical Data Processing (Blackwell, Oxford, 1985),
pp. 194�V207

\bibitem{Q_Greene}
R.R. Greene, The rational approximation to the acoustic wave equation with bottom interaction.
J. Acoust. Soc. Am. 76, 1764�V1773 (1984)

\bibitem{Q_Pade}
A. Bamberger, B. Engquist, L. Halpern, P. Joly, Higher order parabolic wave equation approximations
in heterogeneous media. SIAM J. Appl. Math. 48, 129�V154 (1988)

\bibitem{AESD}
H.K. Brock, The AESD parabolic equation model. Rep. TN-12. Naval Ocean Research and
Development Activity, Stennis Space Center, MS, 1978

\bibitem{B_twe}
P.G. Bergman, The wave equation in a medium with a variable index of refraction. J. Acoust.
Soc. Am. 17, 329�V333 (1946)

\bibitem{a_a_t_h}
Absorption of Sound in Air versus Humidity and Temperature
Cyril M. Harris (1966)

\end{thebibliography}

\end{document}



















